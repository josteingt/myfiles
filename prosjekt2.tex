\documentclass[a4paper, 12pt]{article}

\usepackage{amsmath}
\usepackage{listings}
\usepackage[utf8]{inputenc}
\usepackage{placeins}
\usepackage{graphicx}
\usepackage{algpseudocode}

\title{Prosjekt 2 FYS3150 H13}
\date{}
\author{Jostein Granheim Trøyflat}



\begin{document}
\maketitle

\begin{itemize}

\item[a]

\item[b]
For $\rho_max = 5$ er det nødvendig med ~160 steg for å få de fire første sifferene riktige, i de tre laveste egenverdiene. For større $\rho_max$ trengs det flere steg for å oppnå det samme, for en $\rho_max = 10$ trengs det ~300 steg. 
\item[c]

\begin{table}[h]
\begin{center}
	\caption{Energi for grunntilstand ved forskjellige harmoniske potensial}
	\label{tab:relativfeil}
	\begin{tabular}{| l | r |}
		\hline
		$\omega$ & Energi \\ \hline
		0.01 & 0.84 \\ \hline
		0.5 & 2.23 \\ \hline
		1 & 4.06 \\ \hline
		5 & 17.44 \\ \hline
	\end{tabular}
	\end{center}
\end{table}

\item[d]

\item[e]

\end{itemize}
\end{document}